\documentclass[sigplan,table]{acmart}

\usepackage{booktabs} % For formal tables
\usepackage{xcolor}

% Copyright
%\setcopyright{none}
%\setcopyright{acmcopyright}
%\setcopyright{acmlicensed}
\setcopyright{rightsretained}
%\setcopyright{usgov}
%\setcopyright{usgovmixed}
%\setcopyright{cagov}
%\setcopyright{cagovmixed}

\def\tightlist{\itemsep1pt\parskip0pt\parsep0pt}

% DOI
%\acmDOI{10.475/123_4}

% ISBN
%\acmISBN{123-4567-24-567/08/06}

%Conference
\acmConference[Erlang Workshop 2018]{17th ACM SIGPLAN Erlang Workshop}{September 2018}{St. Louis, MO USA}
\acmYear{2018}
\copyrightyear{2018}

%\acmPrice{15.00}

%\acmBadgeL[http://ctuning.org/ae/ppopp2016.html]{ae-logo}
%\acmBadgeR[http://ctuning.org/ae/ppopp2016.html]{ae-logo}


\begin{document}
\title{Hastega}
%\titlenote{Produces the permission block, and
%  copyright information}
\subtitle{Hyper Accelerator of Spreading Tasks for Elixir with GPU Activation}
%\subtitlenote{The full version of the author's guide is available as
%  \texttt{acmart.pdf} document}

\author{Susumu YAMAZAKI}
%\authornote{Dr.~Trovato insisted his name be first.}
%\orcid{1234-5678-9012}
\affiliation{%
  \institution{University of Kitakyushu}
  \streetaddress{1-1 Hibikino, Wakamatsu-ku}
  \city{Kitakyushu}
  \state{Fukuoka}
  \country{Japan}
  \postcode{808-0135}
}
\email{zacky@kitakyu-u.ac.jp}

\author{Masakazu MORI}
%\authornote{The secretary disavows any knowledge of this author's actions.}
\affiliation{%
  \institution{Delight Systems Co., Ltd.}
  \streetaddress{4-34 Izaki, Chuo-ku}
  \city{Fukuoka}
  \state{Fukuoka}
  \country{Japan}
  \postcode{810-0067}
}
\email{mori@delightsystems.com}

\author{Yoshihiro UENO}
%\authornote{This author is the
%  one who did all the really hard work.}
\affiliation{%
  \institution{Delight Systems Co., Ltd.}
  \streetaddress{4-34 Izaki, Chuo-ku}
  \city{Fukuoka}
  \state{Fukuoka}
  \country{Japan}
  \postcode{810-0067}
}
\email{delightadmin@delightsystems.com}

\author{Hideki TAKASE}
\affiliation{%
  \institution{Kyoto University}
  \streetaddress{}
  \city{Kyoto}
  \state{Kyoto}
  \country{Japan}
  \postcode{810-0067}
}
\email{takase@i.kyoto-u.ac.jp}

% The default list of authors is too long for headers.
\renewcommand{\shortauthors}{S. Yamazaki et al.}


\begin{abstract}
We have succeeded in implementing a demonstration program in which an Elixir code invokes directly a GPGPU benchmark by Rustler. We have conducted the performance evaluation of the experimental implementation of GPGPU by Elixir and Rustler. We have got the following result: (1) Elixir and Rustler code using GPU is 1.76--2.12 times and 2.52--2.7 times faster than pure Elixir and Python code executed by only CPU, respectively. (2) Elixir and Rustler code using GPU is almost as fast as Python code using GPU. (3) Native code using GPU is 3.54--5.66 times faster than Elixir and Rustler code and Python code using GPU. This is the potential of optimization. We realize that Erlang VM is not enough performance for such optimization, including elimination of the conversions between lists and vectors, which is the main reason of the overhead. Thus, We will implement Hastega, which is a new processing system of Elixir, which has a enough ability to drive GPUs and optimize the conversions.
\end{abstract}

%
% The code below should be generated by the tool at
% http://dl.acm.org/ccs.cfm
% Please copy and paste the code instead of the example below.
%
\begin{CCSXML}
<ccs2012>
<concept>
<concept_id>10010147.10010169.10010175</concept_id>
<concept_desc>Computing methodologies~Parallel programming languages</concept_desc>
<concept_significance>500</concept_significance>
</concept>
<concept>
<concept_id>10011007.10011006.10011008.10011009.10010175</concept_id>
<concept_desc>Software and its engineering~Parallel programming languages</concept_desc>
<concept_significance>500</concept_significance>
</concept>
<concept>
<concept_id>10011007.10011006.10011008.10011009.10011012</concept_id>
<concept_desc>Software and its engineering~Functional languages</concept_desc>
<concept_significance>500</concept_significance>
</concept>
</ccs2012>
\end{CCSXML}

\ccsdesc[500]{Computing methodologies~Parallel programming languages}
\ccsdesc[500]{Software and its engineering~Parallel programming languages}
\ccsdesc[500]{Software and its engineering~Functional languages}

\keywords{Elixir, Parallel programming languages, Functional languages, Parallel computing, GPGPU, MapReduce}


\maketitle

\input{description}

\bibliographystyle{ACM-Reference-Format}
\bibliography{reference}

\end{document}
