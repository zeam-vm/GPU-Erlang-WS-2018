\documentclass[sigplan]{acmart}

\usepackage{booktabs} % For formal tables


% Copyright
%\setcopyright{none}
%\setcopyright{acmcopyright}
%\setcopyright{acmlicensed}
\setcopyright{rightsretained}
%\setcopyright{usgov}
%\setcopyright{usgovmixed}
%\setcopyright{cagov}
%\setcopyright{cagovmixed}


% DOI
%\acmDOI{10.475/123_4}

% ISBN
%\acmISBN{123-4567-24-567/08/06}

%Conference
\acmConference[Erlang Workshop 2018]{17th ACM SIGPLAN Erlang Workshop}{September 2018}{St. Louis, Missouri USA}
\acmYear{2018}
\copyrightyear{2018}

%\acmPrice{15.00}

%\acmBadgeL[http://ctuning.org/ae/ppopp2016.html]{ae-logo}
%\acmBadgeR[http://ctuning.org/ae/ppopp2016.html]{ae-logo}


\begin{document}
\title{Elixir Performs GPGPU}
%\titlenote{Produces the permission block, and
%  copyright information}
%\subtitle{}
%\subtitlenote{The full version of the author's guide is available as
%  \texttt{acmart.pdf} document}

\author{Susumu YAMAZAKI}
%\authornote{Dr.~Trovato insisted his name be first.}
%\orcid{1234-5678-9012}
\affiliation{%
  \institution{University of Kitakyushu}
  \streetaddress{1-1 Hibikino, Wakamatsu-ku}
  \city{Kitakyushu}
  \state{Fukuoka}
  \country{Japan}
  \postcode{808-0135}
}
\email{zacky@kitakyu-u.ac.jp}

\author{Masakazu MORI}
%\authornote{The secretary disavows any knowledge of this author's actions.}
\affiliation{%
  \institution{Delight Systems Co., Ltd.}
  \streetaddress{4-34 Izaki, Chuo-ku}
  \city{Fukuoka}
  \state{Fukuoka}
  \country{Japan}
  \postcode{810-0067}
}
%\email{}

\author{Yoshihiro UENO}
%\authornote{This author is the
%  one who did all the really hard work.}
\affiliation{%
  \institution{Delight Systems Co., Ltd.}
  \streetaddress{4-34 Izaki, Chuo-ku}
  \city{Fukuoka}
  \state{Fukuoka}
  \country{Japan}
  \postcode{810-0067}
}
%\email{larst@affiliation.org}

\author{Hideki TAKASE}
\affiliation{%
  \institution{Kyoto University}
  \streetaddress{}
  \city{Kyoto}
  \state{Kyoto}
  \country{Japan}
  \postcode{810-0067}
}


% The default list of authors is too long for headers.
\renewcommand{\shortauthors}{S. Yamazaki et al.}


\begin{abstract}
We have succeeded in implementing a demonstration program in which an Elixir code invokes directly a GPGPU benchmark by Rustler. 
\end{abstract}

%
% The code below should be generated by the tool at
% http://dl.acm.org/ccs.cfm
% Please copy and paste the code instead of the example below.
%
\begin{CCSXML}
<ccs2012>
 <concept>
  <concept_id>10010520.10010553.10010562</concept_id>
  <concept_desc>Computer systems organization~Embedded systems</concept_desc>
  <concept_significance>500</concept_significance>
 </concept>
 <concept>
  <concept_id>10010520.10010575.10010755</concept_id>
  <concept_desc>Computer systems organization~Redundancy</concept_desc>
  <concept_significance>300</concept_significance>
 </concept>
 <concept>
  <concept_id>10010520.10010553.10010554</concept_id>
  <concept_desc>Computer systems organization~Robotics</concept_desc>
  <concept_significance>100</concept_significance>
 </concept>
 <concept>
  <concept_id>10003033.10003083.10003095</concept_id>
  <concept_desc>Networks~Network reliability</concept_desc>
  <concept_significance>100</concept_significance>
 </concept>
</ccs2012>
\end{CCSXML}

\ccsdesc[500]{Computer systems organization~Embedded systems}
\ccsdesc[300]{Computer systems organization~Redundancy}
\ccsdesc{Computer systems organization~Robotics}
\ccsdesc[100]{Networks~Network reliability}


\keywords{ACM proceedings, \LaTeX, text tagging}


\maketitle

\input{description}

\bibliographystyle{ACM-Reference-Format}
\bibliography{reference}

\end{document}
